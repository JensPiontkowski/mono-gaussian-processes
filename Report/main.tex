\documentclass[twoside,11pt]{article}

% Any additional packages needed should be included after jmlr2e.
% Note that jmlr2e.sty includes epsfig, amssymb, natbib and graphicx,
% and defines many common macros, such as 'proof' and 'example'.
%
% It also sets the bibliographystyle to plainnat; for more information on
% natbib citation styles, see the natbib documentation, a copy of which
% is archived at http://www.jmlr.org/format/natbib.pdf

\usepackage{jmlr2e}
\usepackage{amsmath}
\usepackage{graphicx}
\usepackage{wrapfig}
\graphicspath{ {./images/} }
% Definitions of handy macros can go here

\newcommand{\dataset}{{\cal D}}
\newcommand{\fracpartial}[2]{\frac{\partial #1}{\partial  #2}}

% Heading arguments are {volume}{year}{pages}{submitted}{published}{author-full-names}

%\jmlrheading{1}{2000}{1-48}{4/00}{10/00}{Marina Meil\u{a} and Michael I. Jordan}

% Short headings should be running head and authors last names

%\ShortHeadings{Learning with Mixtures of Trees}{Meil\u{a} and Jordan}
\firstpageno{1}

%%%%%%%%%%%% Custom Commands
\newcommand{\X}{\mathbf{X}}
\newcommand{\I}{\mathbf{I}}
\newcommand{\Y}{\mathbf{Y}}





\begin{document}

\title{Unimodality as an extention of Monotonicity in Gaussian Processes}

\author{\name Author \email emailid@aalto.fi \\
       Aalto University
       }

%\editor{Leslie Pack Kaelbling}

\maketitle

\begin{abstract}%   <- trailing '%' for backward compatibility of .sty file
\emph{Dummy abstract!!}
In probability theory and statistics, a Gaussian process is a stochastic process (a collection of random variables indexed by time or space), such that every finite collection of those random variables has a multivariate normal distribution, i.e. every finite linear combination of them is normally distributed. The distribution of a Gaussian process is the joint distribution of all those (infinitely many) random variables, and as such, it is a distribution over functions with a continuous domain, e.g. time or space. A machine-learning algorithm that involves a Gaussian process uses lazy learning and a measure of the similarity between points (the kernel function) to predict the value for an unseen point from training data. The prediction is not just an estimate for that point, but also has uncertainty information—it is a one-dimensional Gaussian distribution (which is the marginal distribution at that point)\cite{CaCh06,SoMuLe03}.
\end{abstract}

\begin{keywords}
  Gaussian Processes, Informative Priors, Unimodality 
\end{keywords}

\section{Introduction}
Gaussian processes are probabilistic models which offer a non parameteric fully bayesian framework for learning a regression task. The prior information is usually encoded within the choice of the mean and covariance functions along with the hyperparameters of these function. The prior choice of monotonicity constraint was shown to be enforcable with the use of psuedo inputs, Gaussian process derivatives and using a sigmoidal link function to enforce the derivatives of a given sign \cite{JaaAki10}. In this project we try extend the monotonicity constraint to enforce a unimodality constraint. 


\section{Related Works}
\subsection{Gaussian Processes}
We can model a Gaussian process regression as a stochastic process with input $X$, evaluating to the underlying latent function $f$, to which the noise variance is added to form the obseved output $Y$.
\begin{align*}
(\Y|\X) 	&\sim p(\Y|f) p(f|\X)\\
	   		&\sim \mathcal{N}\big(0,\sigma^2\I\big) \mathcal{N}\big(m(\X),k(\X,\X)\big)\\
	   		&\sim  \mathcal{N}\big(m(\X),k(\X,\X)+\sigma^2\I\big)
\end{align*} 
To make predictions $f^*$ for new input points $X^*$ we have the following joint distribution,

\begin{align*}
\begin{bmatrix} \Y \\ f^* \end{bmatrix}
 \sim
 \mathcal{N}
 \begin{pmatrix}
  0,
  \begin{bmatrix}
  	K(\X,\X) & K(\X,\X^*)\\
  	K(\X^*,\X) & K(\X^*,\X^*)
  \end{bmatrix}
 \end{pmatrix} 
\end{align*}

The conditional distribution of the prediction follows the normal form,
\begin{align*}
f^*|\X^*,\X,f \sim \mathcal{N}\Big( & K(\X^*,\X)\big(K(\X,\X)+\sigma^2\I\big)^{-1}\Y,\\
								  & K(\X^*,\X^*)-K(\X^*,\X)\big(K(\X,\X)+\sigma^2\I\big)^{-1}K(\X,\X^*)\Big)
\end{align*}

\subsection{Gaussian Process derivatives}
Differentiation is a linear operator due to which the derivative of a GP also remains gaussian. The derivative information can be hence be incorporated into the GP model. The RBF covariance function incorporating the derivative information is has the form,
\begin{align*}
Cov\big[f^{(i)},f^{(j)}\big] = \eta^2 exp&\bigg(-\frac{1}{2}\sum_{d=1}^D \rho_d^{-2}\big(x_d^{(i)}-x_d^{(j)}\big)^2\bigg)\\
Cov\Bigg[\frac{\partial f^{(i)}}{\partial x_g^{(i)}},f^{(j)}\Bigg] = \eta^2 exp&\Bigg(-\frac{1}{2}\sum_{d=1}^{D}\rho_d^{-2}\big(x_d^{(i)}-x_d^{(j)}\big)^2\Bigg)\bigg(-\rho_g^{-2}\big(x_g^{(i)}-x_g^{(j)}\big)\bigg) \\
Cov\Bigg[\frac{\partial f^{(i)}}{\partial x_g^{(i)}},\frac{\partial f^{(j)}}{\partial x_h^{(j)}}\Bigg] = \eta^2 exp&\Bigg(-\frac{1}{2}\sum_{d=1}^{D}\rho_d^{-2}\big(x_d^{(i)}-x_d^{(j)}\big)^2\Bigg) \\ 
& \rho_g^{-2}\bigg(\delta_{gh}-\rho_h^{-2}\big(x_h^{(i)}-x_h^{(j)}\big)\big(x_g^{(i)}-x_g^{(j)}\big)\bigg)
\end{align*}

\subsection{Monotonicity using derivative information}
Using the derivative information we can enforce a monotonicity constraint by using sigmoidal likelihood for the derivative observations. A set of M points($\X_{\partial}$) over the input space are choosen and monotonicity constraint is enforced over those points instead of evaluating the derivative over the whole input space. 
\begin{align}
p\Bigg(\begin{bmatrix} f \\ f_{\partial} \end{bmatrix} \Bigg| \begin{bmatrix} \Y \\ \Y_{\partial}\end{bmatrix} \Bigg) = 
\frac{1}{C}\ p\Bigg(\begin{bmatrix} f \\ f_{\partial}\end{bmatrix} \Bigg| \begin{bmatrix} \X \\ \X_{\partial}\end{bmatrix} \Bigg) p(\Y|f)p(\Y_{\partial}|f_{\partial}) \label{jointdens}
\end{align}
The last probability term acts as the derviative likelihood driving function values without monotonicity to a low probability. The derivative likelihood has the form,
\begin{align}
p(\Y_{\partial}|f_{\partial}) = \prod_{i=1}^{M}\phi\bigg(m f_{\partial}^{(i)} \frac{1}{v}\bigg) \label{linkfunc}
\end{align}
where M is the number of psuedo derivative points, $\phi$ is a sigmoidal link function, $m$ is the latent derivative function which gives us the sign of dervivative that we are trying to enforce and the parameter $v$ controls the steepness of the sigmoidal link function.



\section{Unimodality constraint using Monotonicity}
The latent derivative function $m$ in equation \ref{linkfunc} can be modeled as an input dependant function which can be used to enforce shape constraints. The unimodality information can be modelled by a using parameteric monotonic function to represent the derivative information $m$. The primary role of the monotonic derivative function model would be to learn the mode of the data accurately, where the sign of the derivative would flip. 

We experimented with three different models for the latent derivative function:
\begin{enumerate}
	\item Linear model of the form $m(x)=ax+b$
	\item A zero mean Gaussian process of the form $m(x)=GP(0,k_m(x,x))$
	\item A linear mean Gaussian process $m(x)=GP(ax+b,k_m(x,x))$ 
\end{enumerate}







\section{Experiments}
To




% Acknowledgements should go at the end, before appendices and references









%\acks{We would like to acknowledge support for this project
%from the National Science Foundation (NSF grant IIS-9988642)
%and the Multidisciplinary Research Program of the Department
%of Defense (MURI N00014-00-1-0637). }
%
%% Manual newpage inserted to improve layout of sample file - not
%% needed in general before appendices/bibliography.
%
%\newpage
%
%\appendix
%\section*{Appendix A.}
%\label{app:theorem}
%
%% Note: in this sample, the section number is hard-coded in. Following
%% proper LaTeX conventions, it should properly be coded as a reference:
%
%%In this appendix we prove the following theorem from
%%Section~\ref{sec:textree-generalization}:
%
%In this appendix we prove the following theorem from
%Section~6.2:
%
%\noindent
%{\bf Theorem} {\it Let $u,v,w$ be discrete variables such that $v, w$ do
%not co-occur with $u$ (i.e., $u\neq0\;\Rightarrow \;v=w=0$ in a given
%dataset $\dataset$). Let $N_{v0},N_{w0}$ be the number of data points for
%which $v=0, w=0$ respectively, and let $I_{uv},I_{uw}$ be the
%respective empirical mutual information values based on the sample
%$\dataset$. Then
%\[
%	N_{v0} \;>\; N_{w0}\;\;\Rightarrow\;\;I_{uv} \;\leq\;I_{uw}
%\]
%with equality only if $u$ is identically 0.} \hfill\BlackBox
%
%\noindent
%{\bf Proof}. We use the notation:
%\[
%P_v(i) \;=\;\frac{N_v^i}{N},\;\;\;i \neq 0;\;\;\;
%P_{v0}\;\equiv\;P_v(0)\; = \;1 - \sum_{i\neq 0}P_v(i).
%\]
%These values represent the (empirical) probabilities of $v$
%taking value $i\neq 0$ and 0 respectively.  Entropies will be denoted
%by $H$. We aim to show that $\fracpartial{I_{uv}}{P_{v0}} < 0$....\\
%
%{\noindent \em Remainder omitted in this sample. See http://www.jmlr.org/papers/ for full paper.}
%
%
%\vskip 0.2in

\newpage
\bibliography{references}

\end{document}